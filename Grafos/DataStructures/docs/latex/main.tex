%%%%%%%%%%%%%%%%%%%%%%%%%%%%%%%%%%%%%%%%%%%%%%%%%%%%%%%%%%%%%%%%%%%%%%
% How to use writeLaTeX: 
%
% You edit the source code here on the left, and the preview on the
% right shows you the result within a few seconds.
%
% Bookmark this page and share the URL with your co-authors. They can
% edit at the same time!
%
% You can upload figures, bibliographies, custom classes and
% styles using the files menu.
%
%%%%%%%%%%%%%%%%%%%%%%%%%%%%%%%%%%%%%%%%%%%%%%%%%%%%%%%%%%%%%%%%%%%%%%

\documentclass[12pt]{article}

\usepackage{sbc-template}

\usepackage{graphicx,url}

%\usepackage[brazil]{babel}
\usepackage[utf8]{inputenc}

\usepackage{minted}

\sloppy

\title{Implementação de Estruturas de Dados em C++}

\author{Lucas Gualtieri F. E.\inst{1} }

\address{
Pontifical Catholic University of Minas Gerais\\
Coração Eucarístico -- 30535-901 -- Belo Horizonte -- MG -- Brazil
\email{lgualtieri@sga.pucminas.br}
}

\begin{document} 

\maketitle

\begin{resumo}
Este trabalho apresenta a implementação de diversas estruturas de dados, desenvolvidas em C++ utilizando templates para que as mesmas fossem genéricas. As estruturas desenvolvidas incluem:

\begin{itemize}
\item Lista
\item Fila
\item Pilha
\item Matriz
\end{itemize}

Além das implementações genéricas, uma matriz de inteiros foi implementada, conforme especificado nos requisitos do projeto. Para validar as estruturas implementadas, foram criados arquivos de teste para garantir o correto funcionamento de todas as operações de cada estrutura.
\end{resumo}

\newpage % Inicia uma nova página

\section{Implementação} \label{sec:implementacao}
Nesta seção, detalho a implementação das estruturas de dados desenvolvidas. As estruturas foram divididas em dois grupos principais: \textit{lineares} e \textit{flexíveis}, como ilustrado na Tabela \ref{tab:estruturas}.

\begin{table}[h!]
\centering
\begin{tabular}{|c|c|}
    \hline
    \textbf{Lineares} & \textbf{Flexíveis} \\ \hline
    \texttt{LinearList<T>} &  \texttt{LinkedList<T>} \\ \hline
    \texttt{LinearStack<T>} & \texttt{DoublyLinkedList<T>} \\ \hline
    \texttt{LinearQueue<T>} & \texttt{LinkedStack<T>} \\ \hline
    \texttt{Matrix<T>} &      \texttt{LinkedQueue<T>} \\ \hline
    \texttt{MatrixInt} & \\ \hline
\end{tabular}
\caption{Estruturas de dados desenvolvidas.}
\label{tab:estruturas}
\end{table}

Cada uma dessas estruturas foi projetada para atender a diferentes necessidades, mas todas compartilham uma interface comum que garante consistência e reusabilidade do código. 

As estruturas lineares, com exceção da \texttt{Matrix<T>} e \texttt{MatrixInt}, possuem métodos adicionais relacionados à manipulação da capacidade de memória reservada. Abaixo, são apresentados esses métodos:

\begin{minted}[linenos, frame=single, fontsize=\footnotesize]{cpp}
void reserve(size_t newCapacity) {
    if (newCapacity > maxSize) {
        resize(newCapacity);
    }
}

void shrink_to_fit() {
    resize(this->_size);
}

size_t capacity() {
    return maxSize;
}
\end{minted}

\newpage % Inicia uma nova página

\subsection{Listas}

Nesta subseção, discuto as três variações de listas que foram implementadas: a lista linear, a lista simplesmente encadeada e a lista duplamente encadeada. Embora essas estruturas tenham diferentes organizações internas, todas compartilham uma interface comum chamada \texttt{List<T>}. Essa interface define um conjunto de operações básicas que garantem a consistência na manipulação das listas. A seguir, apresento a interface \texttt{List<T>}:

\begin{minted}[linenos, frame=single, fontsize=\footnotesize]{cpp}
template <typename T>
class List {
  protected:
    size_t _size;
    
  public:
    virtual ~List() = default;
    
    virtual void push_front(T value) = 0;
    virtual void push_back(T value) = 0;
    virtual T pop_front() = 0;
    virtual T pop_back() = 0;
    virtual void add(const T& value, unsigned int pos) = 0;
    virtual T remove(unsigned int pos) = 0;
    virtual T& front() const = 0;
    virtual T& back() const = 0;
    virtual void sort() = 0;
    
    virtual bool contains(const T& value) const = 0;
    virtual void clear() final { while (!empty()) pop_front(); }
    virtual size_t size() const final { return _size; }
    virtual bool empty() const final { return _size == 0; }
}
\end{minted}

\newpage % Inicia uma nova página

\subsection{Filas}

Nesta subseção, discuto as duas variações de filas que foram implementadas: a fila linear e a fila simplesmente encadeada. Embora essas estruturas tenham diferentes organizações internas, todas compartilham uma interface comum chamada \texttt{Queue<T>}. Essa interface define um conjunto de operações básicas que garantem a consistência na manipulação das filas. A seguir, apresento a interface \texttt{Queue<T>}:

\begin{minted}[linenos, frame=single, fontsize=\footnotesize]{cpp}
template <typename T>
class Queue {
  protected:
	size_t _size;

  public:
	virtual ~Queue() = default;

	virtual void push(const T& value) = 0;
	virtual T pop() = 0;
	virtual T& peek() const = 0;

	virtual bool contains(const T& value) const = 0;
	virtual void clear() { while (!empty()) pop(); }
	virtual size_t size() const final { return _size; }
	virtual bool empty() const final { return _size == 0; }
	virtual std::string str() const = 0;
};
\end{minted}

\newpage % Inicia uma nova página

\subsection{Pilhas}

Nesta subseção, discuto as duas variações de pilhas que foram implementadas: a pilha linear e a pilha simplesmente encadeada. Embora essas estruturas tenham diferentes organizações internas, todas compartilham uma interface comum chamada \texttt{Stack<T>}. Essa interface define um conjunto de operações básicas que garantem a consistência na manipulação das pilhas. A seguir, apresento a interface \texttt{Stack<T>}:

\begin{minted}[linenos, frame=single, fontsize=\footnotesize]{cpp}
template <typename T>
class Stack {
  protected:
	size_t _size;

  public:
	virtual ~Stack() = default;

	virtual void push(const T& value) = 0;
	virtual T pop() = 0;
	virtual T& peek() const = 0;

	virtual bool contains(const T& value) const = 0;
	virtual void clear() { while (!empty()) pop(); }
	virtual size_t size() const final { return _size; }
	virtual bool empty() const final { return _size == 0; }
	virtual std::string str() const = 0;
};
\end{minted}

\newpage % Inicia uma nova página

\subsection{Matrizes}

Nesta subseção, discuto as duas variações de matrizes que foram implementadas: a matriz genérica e a matriz de inteiros. A matriz genérica foi implementada usando uma \texttt{vector<vector<T>>}. Já a matriz de inteiros usou um simples \texttt{int**}. A seguir, apresento os métodos de ambas as classes:

\begin{minted}[linenos, frame=single, fontsize=\footnotesize]{cpp}
template <typename T>
class Matrix {
    std::vector<std::vector<T>> matrix;
  public:
    const int height, width;

    Matrix(int height, int width);
    Matrix(std::initializer_list<std::initializer_list<T>> list);
    
    bool inBounds(int i, int j);
    bool notInBounds(int i, int j);
    
    std::vector<T>& operator[](int i);
    
    std::string str() const;
    std::string str(bool flag) const;
    
    friend ostream& operator<<ostream& os, const Matrix<T>& m);
    
    class Iterator;
    Iterator begin();
    Iterator end();
};
\end{minted}

\begin{minted}[linenos, frame=single, fontsize=\footnotesize]{cpp}
class Matrix {
    int** matrix;
  public:
    const int height, width;

    Matrix(int height, int width);
    
    bool inBounds(int i, int j);
    bool notInBounds(int i, int j);
    
    int* operator[](int i);
    
    std::string str() const;
    std::string str(bool flag) const;
    
    friend ostream& operator<<ostream& os, const Matrix<T>& m);
};
\end{minted}

\newpage % Inicia uma nova página
\section{Estrutura do Projeto}
Os exemplos de uso das estruturas se encontra nos arquivos de teste
\begin{verbatim}
.
|-- bin
|   |-- doublyLinkedListTest
|   |-- linearListTest
|   |-- linearQueueTest
|   |-- linearStackTest
|   |-- linkedListTest
|   |-- linkedQueueTest
|   |-- linkedStackTest
|   |-- matrixTest
|-- docs
|   |-- ideias.txt
|-- include
|   |-- cell.hpp
|   |-- list
|   |   |-- doublyLinkedList.hpp
|   |   |-- linearList.hpp
|   |   |-- linkedList.hpp
|   |   |-- list.hpp
|   |-- matrix
|   |   |-- matrix.hpp
|   |-- queue
|   |   |-- linearQueue.hpp
|   |   |-- linkedQueue.hpp
|   |   |-- queue.hpp
|   |-- stack
|       |-- linearStack.hpp
|       |-- linkedStack.hpp
|       |-- stack.hpp
|-- tests
|   |-- doublyLinkedListTest.cc
|   |-- linearListTest.cc
|   |-- linearQueueTest.cc
|   |-- linearStackTest.cc
|   |-- linkedListTest.cc
|   |-- linkedQueueTest.cc
|   |-- linkedStackTest.cc
|   |-- matrixTest.cc
|-- util
|-- util.hpp
\end{verbatim}

\end{document}
