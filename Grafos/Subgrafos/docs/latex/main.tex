%%%%%%%%%%%%%%%%%%%%%%%%%%%%%%%%%%%%%%%%%%%%%%%%%%%%%%%%%%%%%%%%%%%%%%
% How to use writeLaTeX: 
%
% You edit the source code here on the left, and the preview on the
% right shows you the result within a few seconds.
%
% Bookmark this page and share the URL with your co-authors. They can
% edit at the same time!
%
% You can upload figures, bibliographies, custom classes and
% styles using the files menu.
%
%%%%%%%%%%%%%%%%%%%%%%%%%%%%%%%%%%%%%%%%%%%%%%%%%%%%%%%%%%%%%%%%%%%%%%

\documentclass[12pt]{article}

\usepackage{sbc-template}

\usepackage{graphicx,url}

%\usepackage[brazil]{babel}
\usepackage[utf8]{inputenc}

\usepackage{minted}

\usepackage[utf8]{inputenc}
\usepackage{amsmath}

\sloppy

\title { Implementação de um Gerador de Subgrafos \\ de um Grafo Completo em C++ }

\author {Lucas Gualtieri F. E.\inst{1}, Gabriel Quaresma \inst{1} }

\address{
    Pontifical Catholic University of Minas Gerais\\
    Coração Eucarístico -- 30535-901 -- Belo Horizonte -- MG -- Brazil
    \email{\{lgualtieri, gabrielquaresma\}@sga.pucminas.br}
}

\begin{document} 

\maketitle

\begin{resumo}
    Este documento descreve a implementação de um gerador de subgrafos de um grafo completo com \(n\) vértices, onde \(n\) é informado pelo usuário. A solução foi implementada em C++ utilizando uma lista linear personalizada. O programa gera todos os subgrafos possíveis e informa o número total de subgrafos gerados.
\end{resumo}

\section{Introdução}
    Neste projeto, foi desenvolvido um gerador de subgrafos de um grafo completo com \(n\) vértices. A tarefa foi implementada em C++, e o programa permite ao usuário informar o valor de \(n\), em que o grafo gerado terá \(n\) vértices e todas as arestas possíveis. O objetivo é gerar todos os subgrafos possíveis e calcular o número total de subgrafos gerados.

\section{Metodologia}
    A implementação foi realizada utilizando C++ e fez uso de uma lista linear personalizada para armazenar os conjuntos de vértices e arestas. A função principal do programa é a geração do conjunto potência (\textit{PowerSet}) dos vértices e arestas, a partir dos quais são gerados os subgrafos.

\subsection{Estruturas de Dados}
O código faz uso de duas estruturas de dados principais:
\begin{itemize}
    \item \textbf{Vertices}: Uma lista linear de inteiros representando os vértices de um subgrafo.
    \item \textbf{Arestas}: Uma lista linear de listas lineares de inteiros, representando as arestas entre os vértices do subgrafo.
\end{itemize}

\subsection{Funções}
\begin{itemize}
    \item \textbf{factorial(int n)}: Calcula o fatorial de \(n\).
    \item \textbf{subsetCombinations(int i, int n)}: Calcula o número de combinações possíveis de \(i\) elementos a partir de um conjunto de \(n\) elementos.
    \item \textbf{PowerSet(LinearList$<$T$>$ set, size\_t min, size\_t max)}: Gera o conjunto potência (subconjuntos) de um conjunto dado. Permite especificar o tamanho mínimo e máximo dos subconjuntos gerados.
    \item \textbf{subgraphs(int N)}: Calcula o número total de subgrafos de um grafo completo com \(N\) vértices.
\end{itemize}

\section{Implementação}
A função PowerSet implementada neste trabalho é uma solução original, desenvolvida para atender às necessidades do projeto. Nossa implementação foi projetada para lidar com a definição de tamanhos mínimos e máximos para os subconjuntos gerados. Essa personalização proporciona maior flexibilidade na geração dos subconjuntos, isso permitiu que o processo ficasse mais de perto das ideias de implementação do projeto. A seguir, apresentamos o código completo implementado para o projeto:

\begin{minted}[linenos, frame=single, fontsize=\footnotesize, tabsize=2,breaklines=true]{cpp}
typedef LinearList<int> Vertices;
typedef LinearList<LinearList<int>> Arestas;

size_t subsetCombinations(int i, int n) {
	return factorial(n) / (factorial(i) * factorial(n - i));
}

template <typename T>
LinearList<LinearList<T>> PowerSet(LinearList<T> set, size_t min = 0, size_t max = 0x7fffffff) {

	if (max == 0x7fffffff) max = set.size();

	int N = set.size();

	LinearList<LinearList<T>> powerSet(pow(2, N));

	for (int i = min; i <= max; i++) {

		LinearList<T> subset(i);

		int currentIndex = 0;

		size_t combinations = subsetCombinations(i, N);

		for (int j = 0; j < combinations; j++) {

			while (subset.size() < i) {

				if (currentIndex >= N) {
					if (subset.back() == set.back()) subset.pop_back();
					currentIndex = set.indexOf(subset.pop_back()) + 1;
				}

				subset.push_back(set[currentIndex++]);
			}

			powerSet.push_back(subset);

			if (!subset.empty()) subset.pop_back();
		}
	}

	return powerSet;
}
\end{minted}

\newpage % Inicia uma nova página

\begin{minted}[linenos, frame=single, fontsize=\footnotesize, tabsize=4,breaklines=true, firstnumber=45]{cpp}
size_t subgraphs(int N) {

	size_t number = 0;

	for (int i = 1; i <= N; i++) {
		number += pow(2, (i * i - i) / 2) * subsetCombinations(i, N);
	}

	return number;
}

int main() {

	int N;

	cout << "Digite o |V| do grafo completo: ";
	cin >> N;

	LinearList<int> set(1, N);

	int i = 1;

	cout << endl;

	for (Vertices V : PowerSet(set, 1)) {

		Arestas edgesSet = PowerSet(V, 2, 2);

		for (Arestas E : PowerSet(edgesSet)) {

			cout << "Subgrafo " << i++ << ": " << endl;
			cout << "V = " << V << endl;
			cout << "E = " << E << endl << endl;
		}
	}

	cout << "O número de subgrafos gerados foi: " << subgraphs(N) << endl;
}
\end{minted}

\section{Resultados}
O programa gera todos os subgrafos possíveis de um grafo completo com \(n\) vértices, onde \(n\) é informado pelo usuário. Para cada subgrafo, é exibido o conjunto de vértices e arestas que o compõem.

\section{Conclusão}
A implementação do gerador de subgrafos foi concluída com sucesso. O programa cumpre o objetivo de gerar e listar todos os subgrafos de um grafo completo com \(n\) vértices, além de calcular o número total de subgrafos gerados. A estrutura de dados utilizada facilitou a manipulação dos conjuntos de vértices e arestas, e o uso do conjunto potência foi essencial para a geração dos subgrafos.

\end{document}